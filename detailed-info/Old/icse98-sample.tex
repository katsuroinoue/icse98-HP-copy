\documentstyle[twocolumn,icse98]{article}

\begin{document}

\title{ICSE 98 Conference Proceeedings Format}

\author{
        \hspace*{-2ex}
        \parbox{2.3in} {\begin{center}
	{\authornamefont First Author}\\ 
        First author's affiliation\\
        1st line of address\\
        2nd line of address\\
        last line, including country \\
	Telephone number\\
	First author's email address
	\end{center} }
        \parbox{2.3in} {\begin{center}
        {\authornamefont Second Author}\\
	Advanced Research Group\\
	Kiwi Computers\\
	386 Hard Drive\\
	Mountain Foo, CA  95030\\
	+1 408 555 1212\\
	author2@netaddress
	\end{center} }
        \parbox{2.3in} {\begin{center}
        {\authornamefont Third N. Lastauthor}\\
	Dipartimento di Elettronica\\
	Politecnico di Milano\\
	P.za Leonardo da Vinci, 32\\
	20133 Milano, Italia\\
	+39 2 555 93540\\
	author3@netaddress
	\end{center} }
}


\maketitle
\copyrightspace

\abstract
This is a sample paper using the format and guidelines 
required for the {\it ICSE 98 Conference Proceedings}. It 
includes instructions for preparing a camera-ready copy of 
your accepted submission.

\keywords
Guides, instructions, author's kit, conference publications

\section{INTRODUCTION}
The {\it Proceedings} of ICSE 98 represent the final archival 
records of the conference. To give the book a high quality 
appearance we ask that authors follow these guidelines. In 
essence, we ask you to make your document look as much 
like this document as possible. The easiest way to do this is 
simply to replace the flow content of this file with your own 
material.
 
This FrameMaker document has several defined paragraph 
tags to help you format your text (e.g., Title, AuthorName, 
AuthorAddress, Body, Heading1, Heading2, and Heading3). 
An electronic copy of this file, as well as files for Word and 
LaTEX formats, may be downloaded from the ICSE 98 on-line author kit \cite{EAK}.

\section{PAGE LIMIT AND PAGE SIZE}
Submissions in different categories have different page 
limits that must be adhered to. Papers, for example, should 
be no longer than 11 pages. Submissions that exceed the 
limit for their category will not be reviewed.
 
All material on each page should fit within a rectangle of 
18 x 23.5 cm (7" x 9.25"), centered on the page, beginning 
1.9 cm (.75") from the top of the page, with a .85 cm (.33") 
space between two 8.4 cm (3.3") columns. Use either US 
Letter or A4 paper. Right margins should be justified, not 
ragged.

\section{TYPESET TEXT}
Submissions should be prepared on a typesetter or word 
processor. Please use a 10-point Times Roman font, or other 
Roman font with serifs, as close as possible in appearance to 
Times Roman in which these guidelines have been set. Note 
that different components (such as title, authors, headers -- 
see below) use the same font, but with different sizes and 
styles. The target is to have a 10-point text, as you see here. 
Please do not use sans-serif or non-proportional fonts except 
for special purposes, such as distinguishing source code text 
(e.g., \verb|#include <iostream.h>|). Fonts similar to Times 
Roman include Times, Computer Modern Roman, and Press.
 
If you do not have a laser printer, you may be able to arrange 
for a business to print your document for you. If no laser 
printer is available, then please ask the conference office for 
assistance. 

\subsection{Title and Authors}
The title (18-point bold), authors' names (12-point bold), 
and affiliations (12-point) run across the full width of the 
page -- one column 17.8 cm (7") wide. Please also include 
phone numbers and e-mail addresses. See the top of this 
page for three names with different addresses. Note that each 
of the names/addresses has its own table cell in a table with 
invisible borders. If only one address is needed, center all 
address text in a single-column table. For two addresses, use 
two columns, and so on. For more that three authors, you 
may have to improvise (if necessary, you may place some 
address information in a footnote). 

\subsection{Abstract and Keywords}
Every submission (except summaries of Workshops) should 
begin with an abstract of no more than 100 words, followed 
by a set of keywords. The abstract and keywords should be 
placed in the left column of the first page. The abstract 
should be a concise summary of the work and resulting 
conclusions. Keywords should help readers determine if the 
paper contains topics they are interested in.

\subsection{First Page Copyright Notice}
Remember to leave at least 2.5 cm (1") of blank space at the 
bottom of the left column of the first page only. This space is 
reserved for the copyright notice that will be added during 
final printing.

\subsection{Subsequent Pages}
For pages other than the first page, start at the top of the page 
and continue in double-column format. It is preferable (but 
not required) that the two columns on the last page have 
approximately equal length. This can be accomplished by 
adjusting the length of the left column on the last page.

\subsection{References and Citations}
Use the standard {\it Communications of the ACM} format for 
references -- that is, a numbered list at the end of the 
article, ordered alphabetically by first author, and referenced 
by numbers in brackets \cite{Anderson:Impacts}. See the examples of citations at 
the end of this document. Within this template file, use the 
style named Numbered for the text of your citation; the first 
citation should be of Paragraph Tag Numbered1.

References should be published materials accessible to the 
public. Internal technical reports may be cited {\it only} if they 
are easily accessible (i.e., you can give the address to obtain 
it within your citation) and may be obtained by any reader. 
Proprietary information should {\it not be} cited. Private 
communications should be acknowledged, not referenced 
(e.g., "[Robertson, personal communication]").

\subsection{Page Numbering, Headers and Footers}
Do not include headers or footers in your submission. Page 
numbers should be included in your submission for review. 
Final submission of accepted papers should {\it not} include any 
page numbers; they will be added for you when the 
publications are assembled.

\section{SECTIONS}
The title of a section should be in Times Roman 10-point 
bold in all capitals.

\subsection{Subsections}
The title of subsections should be in Times Roman 10-point 
bold with only the initial letters of each word capitalized. 
For subsections and subsubsections, a word like {\it the} and {\it a} 
is not capitalized unless it is the first word of the heading.

\subsubsection{Subsubsections}
The heading for subsubsections should be in Times Roman 
10-point italic with initial letters of each word capitalized.
 
\section{FIGURES}
Figures should be inserted at the appropriate point in your 
text. Figures may extend over the two columns up to 17.8 
cm (7") if necessary. Black and white photographs (not 
Polaroid prints) may be mounted on the camera-ready paper 
with glue or double-sided tape. (To avoid smudges, attach 
figures by paste or tape applied to their {\it back} surfaces only.)

\section{LANGUAGE, STYLE AND CONTENT}
The written and spoken language of ICSE 98 is English. 
Spelling and punctuation may consistently use any dialect 
of English (e.g., British, Canadian or US). Please write for 
an international audience:
 
\begin{smallitem}
\item Write in a straightforward style. Use simple sentence 
structure. Try to avoid long sentences and complex sentence 
structure. Use semicolons carefully.
 
\item Use common and basic vocabulary (e.g., use the word 
``unusual" rather than the word ``arcane").
 
\item Briefly define or explain all technical terms.
 
\item Explain all acronyms when they first appear in your text 
such as, ``World Wide Web (WWW)"
 
\item Explain ``insider" comments. Be sure that your whole 
audience will understand any reference whose meaning 
you do not explain (e.g., do not assume that everyone 
has used a Macintosh or MS-DOS).
 
\item Avoid or explain puns, jokes, and colloquial language. 
Humor and irony are difficult to translate. 
 
\item Use unambiguous forms for representing culturally 
localized concepts, such as times, dates, and currencies, 
(e.g., ``1-5-96" or ``5/1/96" may mean 5 January or 1 
May, and ``seven o'clock" may mean 7:00 am or 1900).
\end{smallitem}
 
Authors are responsible for ensuring that their work is 
conducted in a professional and ethical manner \cite{Anderson:Impacts}, 
including (but not limited to) fully informed consent of 
participants in studies, protection of personal data 
(e.g., \cite{Mackay:Ethics}), 
and permission to use others' copyrighted materials.

\section{INFORMATION AND QUESTIONS}

For more information, contact the ICSE 98 Office at 
icse-98-publications@ics.uci.edu, or phone +1 714 824 8756.

\acknowledgements

This document has been adapted from the Style Sheet 
defined for CHI 96 by Michael J. Muller, Bonnie Nardi, and 
Michael J. Tauber, and numberous people in the CHI 
community. Their contributions are gratefully acknowledged.


\bibliographystyle{abbrv}
\begin{thebibliography}{1}

\bibitem{Anderson:Impacts}
R.~E. Anderson.
\newblock Social impacts of computing: {C}odes of professional ethics.
\newblock {\em Social Science Computing Review}, 10(2):453--469, (Winter 1992).

\bibitem{EAK}
ICSE 98 {E}lectronic {A}uthor {K}it. {A}vailable at
  $<$http://icse98.aist-nara.ac.jp/pubform/$>$.

\bibitem{Mackay:Ethics}
W.~E. Mackay.
\newblock Ethics, lies and videotape...
\newblock In {\em Proc. CHI'95}, pages 138--145, Denver, CO, May 1995. ACM
  Press.

\end{thebibliography}
\end{document}
